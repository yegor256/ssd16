% SPDX-FileCopyrightText: Copyright (c) 2021 Yegor Bugayenko
% SPDX-License-Identifier: MIT

\documentclass{article}
\usepackage{../ssd}
\newcommand*\thetitle{Patterns}
\newcommand*\thesubtitle{Anti-Patterns and Refactoring}
\begin{document}

\lnTitlePage{6}{16}{LrTBIcFhawI}

\lnPitch{
  \innoQuote
    {gamma1994design}
    {Experienced designers evidently know something inexperienced ones don't. What is it? One thing expert designers know \ul{not} to do is solve every problem from first principles. Rather, they reuse solutions that have worked for them in the past. When they find a good solution, they use it again and again. Such experience is part of what makes them experts.}
}

\lnPitch{
  \pptQuote
    {../faces/paul-graham}
    {When I see patterns in my programs, I consider it a sign of trouble. The shape of a program should reflect only the problem it needs to solve. Any other regularity in the code is a sign, to me at least, that I'm using abstractions that aren't powerful enough---often that I'm generating by hand the expansions of some macro that I need to write.}
    {\emph{Revenge of the Nerds}, Paul Graham}
}

\pptToc

\lnPitch{\pptChapter[Patterns]{Some Patterns}}

\lnPitch{
  Design Patterns and Anti-Patterns, Love and Hate (2016)\par
  \pptQR{https://www.yegor256.com/2016/02/03/design-patterns-and-anti-patterns.html}\par
  36 patterns (22 anti-patterns)
}

\begin{lnSnippet}[adapter.java]
class Database {
  String sql(String q);
}
void echo(Book b) {
  print(b.title());
  print(b.author());
}
class BookInDatabase implements Book {
  private Database d;
  private int id;
  String title() {
    return d.sql("SELECT title FROM book WHERE id=%1", id);
  }
}
\end{lnSnippet}
\lnPitch{
  \pptPinQR{https://www.yegor256.com/2015/02/26/composable-decorators.html}
  \pptSection[Decorator]{Adapter, Facade, Proxy, Decorator, Bridge}
  \footnotesize\ffinput{adapter.java}
}

\begin{lnSnippet}[raii.cpp]
class File {
  std::FILE* h;
public:
  File(const char* name) {
    h = std::fopen(name, "w+");
  }
  ~File() {
    std::fclose(h);
  }
}
void foo() {
  f File("foo.txt");
  // write to f
}
\end{lnSnippet}
\lnPitch{
  \pptPinQR{https://www.yegor256.com/2017/08/08/raii-in-java.html}
  \pptSection[RAII]{Resource Acquisition Is Initialization (RAII)}
  \footnotesize\ffinput{raii.cpp}
}

\lnPitch{\pptChapter[Anti]{Some Anti-Patterns}}

\begin{lnSnippet}[goto.c]
void foo(int a) {
  if (a % 2 == 0) {
    printf("Even!");
    goto exit;
  }
  printf("Odd!");
  exit:
}
void foo(int a) {
  if (a % 2 == 0) {
    printf("Even!");
  } else {
    printf("Odd!");
  }
}
\end{lnSnippet}
\lnPitch{
  \pptSection{GOTO}
  \footnotesize\ffinput{goto.c}
}

\begin{lnSnippet}[magic.rb]
def points
  File.readlines("/data/users.csv") # why here?
    .map { |t| t.split(',', 11) } # what is 11?
    .map { a[7].to_i } # why 7?
    .inject(&:+)
end
\end{lnSnippet}
\lnPitch{
  \pptSection[Numbers]{Magic Numbers}
  \small\ffinput{magic.rb}
}

\begin{lnSnippet}[magic2.rb]
def h2sec(h)
  return h * 60 * 60
end

def h2sec(h)
  seconds_in_minutes = 60
  minutes_in_hours = 60
  return h * seconds_in_minutes * minutes_in_hours
end
\end{lnSnippet}
\lnPitch{
  \pptBanner{Magic Numbers ... Not!}
  \small\ffinput{magic2.rb}
}

\lnPitch{
  \pptSection[God]{God Class}
  \pptPic{0.6}{god-class}
}

\lnPitch{
  \pptSection[Spaghetti]{Spaghetti Code}
  \pptPic{0.4}{spaghetti}
}

\lnPitch{
  \pptSection[Lasagna]{Lasagna and Ravioli}
  \pptPic{0.3}{ravioli}
}

\lnPitch{\pptChapter[Anti-OOP]{Anti-OOP Patterns}}

\lnPitch{
  Anti-Patterns in OOP (2014)\par
  \pptQR{https://www.yegor256.com/2014/09/10/anti-patterns-in-oop.html}\par
  Eleven:
  NULL,
  Utility Classes,
  Mutable Objects,
  Getters and Setters,
  Data Transfer Object (DTO),
  Object-Relational Mapping (ORM),
  Singletons,
  Controllers/Managers/Validators,
  Public Static Methods,
  Class Casting,
  Traits and Mixins.
}

\begin{lnSnippet}[dto.java]
// Getters and Setters: WRONG!
Dog dog = new Dog();
dog.setWeight("23kg");
w = dog.getWeight();

// Smart objects: RIGHT!
Dog dog = new Dog("23kg");
int w = dog.weight();
\end{lnSnippet}
\lnPitch{
  \pptPinQR{https://www.yegor256.com/2014/09/16/getters-and-setters-are-evil.html}
  \pptSection[DTO]{Data Transfer Object (DTO)}
  \pptBanner{Getters and Setters}
  \small\ffinput{dto.java}
}

\begin{lnSnippet}[utility.java]
public class NumberUtils {
  public static int max(int a, int b) {
    return a > b ? a : b;
  }
}
public class Max implements Number {
  private final int a;
  private final int b;
  public Max(int x, int y) { this.a = x; this.b = y; }
  public int intValue() {
    return this.a > this.b ? this.a : this.b;
  }
}
\end{lnSnippet}
\lnPitch{
  \pptPinQR{https://www.yegor256.com/2014/05/05/oop-alternative-to-utility-classes.html}
  \pptSection[Utility]{Utility Class}
  \footnotesize\ffinput{utility.java}
}

\begin{lnSnippet}[singleton.java]
class Database {
  public static Database INSTANCE = new Database();
  private Database() {  /* start */ }
  public java.sql.Connection connect() { /* fetch */ }
}
c = Database.INSTANCE.connect();
class Foo {
  private final Database d;
  void foo() {
    this.d.connect();
  }
}
\end{lnSnippet}
\lnPitch{
  \pptPinQR{https://www.yegor256.com/2016/06/27/singletons-must-die.html}
  \pptSection[Singleton]{Singleton}
  \small\ffinput{singleton.java}
}

\begin{lnSnippet}[orm.java]
// ORM: Wrong!
Post post = new Post();
post.setDate(new Date());
post.setTitle("How to cook an omelette");
session.save(post);

// Objects: RIGHT!
Post post = new Post();
post.setDate(new Date());
\end{lnSnippet}
\lnPitch{
  \pptPinQR{https://www.yegor256.com/2014/12/01/orm-offensive-anti-pattern.html}
  \pptSection[ORM]{Object-Relational Mapping (ORM)}
  \small\ffinput{orm.java}
}

\lnPitch{\pptChapter[Refactorings]{Some Refactorings}}

\lnQuote
  {martin-fowler}
  {Whenever I do refactoring, the first step is always the same. I need to build a solid set of tests for that section of code. The tests are essential because even though I follow refactorings structured to avoid most of the opportunities for introducing bugs, I'm still human and still make mistakes. Thus I need solid tests.}
  {fowler1999refactoring}

\begin{lnSnippet}[extract.rb]
def root(a, b, c)
  d = Math.sqrt(b * b - 4 * a * c)
  r1 = (-b + d) / (2 * a)
  r2 = (-b - d) / (2 * a)
  [r1, r2]
end

def root(a, b, c)
  d = Math.sqrt(b * b - 4 * a * c)
  [r(a, b, d, 1), r(a, b, d, -1)]
end
def r(a, b, d, m)
  (-b + d * m) / (2 * a)
end
\end{lnSnippet}
\lnPitch{
  \pptPin{$x_{1,2} = \dfrac{-b \pm \sqrt{b^2 - 4ac}}{2a}$}
  \pptBanner{Extract Method}
  \footnotesize\ffinput{extract.rb}
}

\lnPitch{\innoBVC}

\lnPitch{
  \begin{multicols}{2}
    \innoBook{feathers2004working}
    \par\columnbreak
    \innoBook{fowler1999refactoring}
  \end{multicols}
}

\lnPitch{
  \begin{multicols}{2}
    \innoBook{gamma1994design}
    \par\columnbreak
    \innoBook{meyers2005effective}
  \end{multicols}
}

\lnPitch{
  \pptBanner{Where to publish:}
  SPLASH: ACM SIGPLAN conference on Systems, Programming, Languages, and Applications\par
  International Conference on Code Quality (ICCQ),\br
  in cooperation with ACM SIGPLAN/SIGSOFT and IEEE
}

\lnPitch{
  \pptBanner{Call to Action:}
  In your application demonstrate the usage of 4+ design patterns.
  Also, perform 4+ refactorings, each one in its own pull request.
}

\lnPitch{
  \pptBanner[orange]{Still unresolved issues:}
  \begin{itemize}
    \item How to \ul{prove} certain patterns are anti-patterns?
    \item How to \ul{find} methods for automated refactoring?
    \item How to \ul{guarantee} validity during refactoring?
    \item How to \ul{mine} patterns from code?
  \end{itemize}
}

\end{document}
