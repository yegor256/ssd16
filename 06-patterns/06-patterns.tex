% (The MIT License)
%
% Copyright (c) 2021 Yegor Bugayenko
%
% Permission is hereby granted, free of charge, to any person obtaining a copy
% of this software and associated documentation files (the 'Software'), to deal
% in the Software without restriction, including without limitation the rights
% to use, copy, modify, merge, publish, distribute, sublicense, and/or sell
% copies of the Software, and to permit persons to whom the Software is
% furnished to do so, subject to the following conditions:
%
% The above copyright notice and this permission notice shall be included in all
% copies or substantial portions of the Software.
%
% THE SOFTWARE IS PROVIDED 'AS IS', WITHOUT WARRANTY OF ANY KIND, EXPRESS OR
% IMPLIED, INCLUDING BUT NOT LIMITED TO THE WARRANTIES OF MERCHANTABILITY,
% FITNESS FOR A PARTICULAR PURPOSE AND NONINFRINGEMENT. IN NO EVENT SHALL THE
% AUTHORS OR COPYRIGHT HOLDERS BE LIABLE FOR ANY CLAIM, DAMAGES OR OTHER
% LIABILITY, WHETHER IN AN ACTION OF CONTRACT, TORT OR OTHERWISE, ARISING FROM,
% OUT OF OR IN CONNECTION WITH THE SOFTWARE OR THE USE OR OTHER DEALINGS IN THE
% SOFTWARE.

\documentclass{article}
\usepackage{../inno}
\usepackage{../slides}
\usepackage{../ssd}
\newcommand*\thetitle{Patterns}
\newcommand*\thesubtitle{Anti-Patterns and Refactoring}
\begin{document}

\plush{\innoTitlePage{6}}

\plush[2]{\innoQuote{books/gamma}{Experienced designers evidently know something inexperienced ones don't. What is it? One thing expert designers know \ul{not} to do is solve every problem from first principles. Rather, they reuse solutions that have worked for them in the past. When they find a good solution, they use it again and again. Such experience is part of what makes them experts.}{\emph{Design Patterns: Elements of Reusable Object-Oriented Software}, Erich Gamma et al.}}

\plush[2]{\innoQuote{people/paul-graham}{When I see patterns in my programs, I consider it a sign of trouble. The shape of a program should reflect only the problem it needs to solve. Any other regularity in the code is a sign, to me at least, that I'm using abstractions that aren't powerful enough---often that I'm generating by hand the expansions of some macro that I need to write.}{\emph{Revenge of the Nerds}, Paul Graham}}

\innoToc

\innoSection[Patterns]{Some Patterns}

\plush{
  Design Patterns and Anti-Patterns, Love and Hate (2016)\par
  \innoQR{https://www.yegor256.com/2016/02/03/design-patterns-and-anti-patterns.html}\par
  36 patterns (22 anti-patterns)
}

\plush[5]{
  \innoPinQR{https://www.yegor256.com/2015/02/26/composable-decorators.html}
  \innoSubsection[Decorator]{Adapter, Facade, Proxy, Decorator, Bridge}
  \innoSnippet[\footnotesize]{adapter.java}
}

\plush[5]{
  \innoPinQR{https://www.yegor256.com/2017/08/08/raii-in-java.html}
  \innoSubsection[RAII]{Resource Acquisition Is Initialization (RAII)}
  \innoSnippet[\footnotesize]{raii.cpp}
}

\innoSection[Anti]{Some Anti-Patterns}

\plush[2]{
  \innoSubsection{GOTO}
  \innoSnippet[\footnotesize]{goto.c}
}

\plush[2]{
  \innoSubsection[Numbers]{Magic Numbers}
  \innoSnippet{magic.rb}
}

\plush[2]{
  \innoBanner{Magic Numbers ... Not!}
  \innoSnippet{magic2.rb}
}

\plush[2]{
  \innoSubsection[God]{God Class}
  \innoPic{0.6}{god-class}
}

\plush[2]{
  \innoSubsection[Spaghetti]{Spaghetti Code}
  \innoPic{0.4}{spaghetti}
}

\plush[2]{
  \innoSubsection[Lasagna]{Lasagna and Ravioli}
  \innoPic{0.3}{ravioli}
}

\innoSection[Anti-OOP]{Anti-OOP Patterns}

\plush[2]{
  Anti-Patterns in OOP (2014)\par
  \innoQR{https://www.yegor256.com/2014/09/10/anti-patterns-in-oop.html}\par
  Eleven:
  NULL,
  Utility Classes,
  Mutable Objects,
  Getters and Setters,
  Data Transfer Object (DTO),
  Object-Relational Mapping (ORM),
  Singletons,
  Controllers/Managers/Validators,
  Public Static Methods,
  Class Casting,
  Traits and Mixins.
}

\plush[3]{
  \innoPinQR{https://www.yegor256.com/2014/09/16/getters-and-setters-are-evil.html}
  \innoSubsection[DTO]{Data Transfer Object (DTO)}
  \innoBanner{Getters and Setters}
  \innoSnippet{dto.java}
}

\plush[3]{
  \innoPinQR{https://www.yegor256.com/2014/05/05/oop-alternative-to-utility-classes.html}
  \innoSubsection[Utility]{Utility Class}
  \innoSnippet[\footnotesize]{utility.java}
}

\plush[3]{
  \innoPinQR{https://www.yegor256.com/2016/06/27/singletons-must-die.html}
  \innoSubsection{Singleton}
  \innoSnippet{singleton.java}
}

\plush[3]{
  \innoPinQR{https://www.yegor256.com/2014/12/01/orm-offensive-anti-pattern.html}
  \innoSubsection[ORM]{Object-Relational Mapping (ORM)}
  \innoSnippet{orm.java}
}

\innoSection[Refactorings]{Some Refactorings}

\plush[2]{\innoQuote{books/fowler-refactoring}{Whenever I do refactoring, the first step is always the same. I need to build a solid set of tests for that section of code. The tests are essential because even though I follow refactorings structured to avoid most of the opportunities for introducing bugs, I'm still human and still make mistakes. Thus I need solid tests.}{\emph{Refactoring: Improving the Design of Existing Code}, Martin Fowler}}

\plush[3]{
  \innoPin{$x_{1,2} = \dfrac{-b \pm \sqrt{b^2 - 4ac}}{2a}$}
  \innoBanner{Extract Method}
  \innoSnippet[\footnotesize]{extract.rb}
}

\innoBVC

\plush{
  \begin{multicols}{2}
    \innoBook{feathers}
      {Michael Feathers}
      {Working Effectively with Legacy Code}
    \par\columnbreak
    \innoBook{fowler-refactoring}
      {Martin Fowler}
      {Refactoring: Improving the Design of Existing Code}
  \end{multicols}
}

\plush{
  \begin{multicols}{2}
    \innoBook{gamma}
      {Erich Gamma et al.}
      {Design Patterns: Elements of Reusable Object-Oriented Software}
    \par\columnbreak
    \innoBook{meyers}
      {Scott Meyers}
      {Effective C++: 55 Specific Ways to Improve Your Programs and Designs}
  \end{multicols}
}

\plush{
  \innoBanner{Where to publish:}
  SPLASH: ACM SIGPLAN conference on Systems, Programming, Languages, and Applications\par
  International Conference on Code Quality (ICCQ),\br
  in cooperation with ACM SIGPLAN/SIGSOFT and IEEE
}

\plush[2]{
  \innoBanner{Call to Action:}
  In your application demonstrate the usage of 4+ design patterns.
  Also, perform 4+ refactorings, each one in its own pull request.
}

\plush[3]{
  \innoBanner[orange]{Still unresolved issues:}
  \begin{itemize}
    \item How to \ul{prove} certain patterns are anti-patterns?
    \item How to \ul{find} methods for automated refactoring?
    \item How to \ul{guarantee} validity during refactoring?
    \item How to \ul{mine} patterns from code?
  \end{itemize}
}

\end{document}
