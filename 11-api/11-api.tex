% SPDX-FileCopyrightText: Copyright (c) 2021 Yegor Bugayenko
% SPDX-License-Identifier: MIT

\documentclass{article}
\usepackage{../ssd}
\newcommand*\thetitle{Microservices}
\newcommand*\thesubtitle{and RESTful APIs}
\begin{document}

\lnTitlePage{11}{16}{SdbzRl8KBYo}

\pptToc

\lnPitch{\pptChapter[SOA]{Service-Oriented Architecture (SOA)}}

\lnPitch{\pptPic{0.9}{soa}}

\lnPitch{
  \pptSection[RPC]{XML RPC}
  \pptPic{0.5}{rpc}
}

\lnPitch{
  \pptSection[SOAP]{Simple Object Access Protocol (SOAP)}
  \pptPic{0.7}{soap}
}

\lnPitch{
  \pptSection[CORBA]{Common Object Request Broker Architecture (CORBA)}
  \pptPic{0.8}{corba}
}

\lnPitch{
  \pptSection[IDL]{Interface Description Language (IDL)}
  \pptPic{0.8}{idl}
}

\lnPitch{
  \pptSection[EJB]{Enterprise Java Beans (EJB)}
  \pptPic{0.5}{ejb}\par
  \footnotesize
  Java Naming and Directory Interface (JNDI) is an API
  that provides naming and directory functionality to applications
  written using the Java.\par
  Remote Method Invocation (RMI) is a Java API that performs
  remote method invocation, the object-oriented equivalent of
  remote procedure calls, with support for direct transfer of
  serialized Java classes and distributed garbage-collection.
}

\lnPitch{\pptChapter[RESTful]{Microservices and RESTful API}}

\lnPitch{
  \pptSection{Microservices}
  \pptPic{0.6}{microservices}\par
  \small
  ``Microservices are a modern interpretation of service-oriented
  architectures used to build distributed software systems." --- Wikipedia
}

\lnPitch{
  \pptBanner{Stateless vs. Stateful Architecture}
  ``A stateless process or application can be understood in isolation. There is no stored knowledge of or reference to past transactions. Each transaction is made as if from scratch for the first time.'' ---
  \href{https://www.redhat.com/en/topics/cloud-native-apps/stateful-vs-stateless}{RedHat}
}

\lnPitch{
  \pptSection[REST]{Representational State Transfer (REST)}
  \pptPic{0.5}{restful}
  \pptPic{0.5}{restful2}
}

\lnPitch{
  \pptSection[HATEOAS]{Hypermedia As The Engine Of Application State (HATEOAS)}
  \pptPic{0.8}{hateoas}
}

\lnPitch{
  \pptSection[Mesh]{Service Mesh}
  \pptPic{0.6}{mesh}\par
  ``A service mesh is a dedicated infrastructure layer for facilitating service-to-service communications between services or microservices, using a proxy'' --- Wikipedia
}

\lnPitch{
  \pptSection{Chatbots}
  \pptPic{0.6}{chatbots}
}

\lnPitch{\pptChapter[gRPC]{Protobuf and gRPC}}

\lnPitch{
  \pptSection{gRPC}
  \pptPic{0.7}{grpc}
}

\lnPitch{
  \pptSection{Protobuf}
  \begin{multicols}{2}
    \pptPic{0.8}{proto1}
    \par\columnbreak
    \pptPic{0.8}{proto2}
  \end{multicols}
}

\lnPitch{\pptPic{0.8}{grpc-joke}}

\lnPitch{\pptChapter[IoT]{Internet of Things (IoT)}}

\lnPitch{\pptPic{0.8}{iot}}

\lnPitch{\innoBVC}

\lnPitch{
  \begin{multicols}{2}
    \innoBook{richardson2013restful}
    \par\columnbreak
    \innoBook{newman2019monolith}
  \end{multicols}
}

\lnPitch{
  \pptBanner{Where to go:}
  IEEE International Conference on Software Architecture (ICSA)
}

\lnPitch{
  \pptBanner{Call to Action:}
  Design your own RESTful API and publish it at rapidapi.com or
  similar place, where APIs are ``published.''
}

\lnPitch{
  \pptBanner[orange]{Still unresolved issues:}
  \begin{itemize}
    \item How to \ul{validate} an API?
    \item How to \ul{generate} an API from object model?
    \item How to \ul{test} API automatically?
    \item How to \ul{spot} integration mistakes between APIs?
  \end{itemize}
}

\end{document}
